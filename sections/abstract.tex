%%
%% Author: Dario Chinelli
%% begin 2019-12-04
%% last mod 2022-02-02
%%

% Preamble
\documentclass{standalone}

% Packages
\usepackage[subpreambles=true]{standalone}
\usepackage{import}

% Document
\begin{document}

The dynamics of pedestrian changes considerably depending on the surrounding space, not just for the intrinsic chaotic movements that people do walking but also due to the reciprocal collisions and environment condition. We have considered some scenarios to implement models and a tool that can give us simulations of the movements of a single pedestrian. To properly simulate pedestrians’ dynamic, we needed to have information about the probability to change direction after every step and in every position of the trajectory. This approach is linked to the path integral in the way that: given a trajectory, it’s possible to say with certain probability where the next step is. This mathematical approach is computationally expensive, even more with the big amount of data we are using. So, we started implementing a discrete system and an easy model; then we moved to more complex one. At the end, we got four types of models in total: two were time dependent and two were independent. From now on we will call the first two: D2Q9 and D2Q9Q9 and TD2Q9 and TD2Q9Q9 the others. The scientific aim has been to create a mathematical framework, inspired to the Lattice-Boltzmann and Cellular Automata, with which we could apprehend, starting from real data, the dynamics of pedestrian and quantify it in terms of lattice transition matrices. The fundamental purpose is to succeed in quantify the probability field, found utilizing different models. This field allows us to study the dynamic and to create simulations of pedestrians and trajectories whose statistics are indistinguishable by construction from the real trajectories’ statistics.


\end{document}
