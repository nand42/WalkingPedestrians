%%
%% Author: Dario Chinelli
%% begin 2019-12-04
%% last mod 2022-02-02
%%

% Preamble
\documentclass{standalone}

% Packages
\usepackage[subpreambles=true]{standalone}
\usepackage{import}

% Document
\begin{document}

\textbf{[ENG]}
The dynamics of pedestrian changes considerably depending on the surrounding space,
not just for the intrinsic chaotic movements that people does walking but also due to the reciprocal collisions and environment condition.
We have considered some scenarios to implement models and a tools that can gives us simulations of the movements of a single pedestrian.
In order to properly simulate a pedestrians' dynamic, is to have information about the probability to change direction after every step, in every positions of the trajectory.
This approach is linked to the path integral in the way that: given a trajectory, it's possible to say with certain probability where the next step is.
This mathematical approach is computationally expensive, even more with the big amount of data we are using.
So we started implementing a discrete system and a easy model and than we moved to more complex model.
In total we get four types of models: two time dependent and tow independent.
From now on we'll call those: D2Q9 and D2Q9Q9 the firsts two;
TD2Q9 and TD2Q9Q9 the others two.

\textbf{[ITA]}
L'obiettivo scientifico è stato creare un metodo, ispirato al Lattice-Boltzman, con cui apprendere, a partire
da dati reali, la dinamica pedonale e quantificarla in termini di matrici di transizione su reticolo.
L'obiettivo fondamentale è riuscire a quantificare il campo di probabilità, trovato utilizzando diversi modelli.
Questo campo ci permette di studiare la dinamica e creare simulazioni di pedoni e traiettorie le cui statistiche
sono indistinguibili per costruzione dalle statistiche delle traiettorie reali

\end{document}
