%%
%% Author: Dario Chinelli
%% begin 2019-12-04
%% last mod 2022-02-02
%%

% Preamble
\documentclass[class=article, crop=false]{standalone}

% Packages
\usepackage[subpreambles=true]{standalone}
\usepackage{import}
\usepackage{graphicx}
\usepackage{amsmath}
\usepackage{subfig}

%	References:
%	> Markov chains  from theory to implementation and experimentation by Gagniuc, Paul A

% Document
\begin{document}\label{chap:MarkovChain}
A \emph{Markov Chain} is a stochastic model.
It describes the future outcome state based on the present state.
In other words, the present state determinate the probabilities for every possible future outcome.
The model's representation is a \emph{stochastic matrix}, from now on called $P$ matrix.
The matrix's entries $P_{ij}$ has as row-index $i$ the starting state and as column-index $j$ the ending state of the system.

\begin{figure}[h]
    \centering
    \subfloat[The diagram of a two-state Markov Chain]{
    \label{diagramMC}
        \includegraphics[ width=0.32\textwidth]{./estratti/Markov Chain scheme two state}
    }\quad\quad
    \subfloat[The transition matrix, also named the Markov matrix]{
    \label{matrixMC}
        \includegraphics[ width=0.4\textwidth]{./estratti/Markov Chain matrix two state}
    }
    \caption{From the Markov diagram to the Markov matrix of a tow-state system}
    \label{MarkovMatrix}
\end{figure}
A two-state Markov chain is the most basic model which can be used for the illustration of the Markov process.
The diagram in (Figure \ref{diagramMC}) represents the possibility that the system has to change from both states.
For instance, from the state $W$ the system can move to the $B$ state with the big black arrow or can remain in the $W$ state with the small white arrow.
The entries in the Markov Matrix in (Figure \ref{matrixMC}) are positives numbers that represent the probability of changing state.






\end{document}
