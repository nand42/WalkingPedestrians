%%
%% Author: Dario Chinelli
%% begin 2019-12-04
%% last mod 2022-02-02
%%

% Preamble
\documentclass[class=article, crop=false]{standalone}

% Packages
\usepackage[subpreambles=true]{standalone}
\usepackage{import}
\usepackage{graphicx}
\usepackage{amsmath}


% Document
\begin{document}

Since the late eighteenth century, theories on human walking and pedestrian movement have been developed from many scientific perspectives \cite{science_walking}. 
Ongoing research has created widespread and diverse knowledge on this subject, branching research into many different specialisms of pedestrian research. 
Over the years, multiple literature review papers \cite{modelling_artificialped,crowdmotion_sota} have become available which aid to create taxonomy in the available literature on pedestrian dynamics.
In 1895, Gustave Le Bon stated in \cite{LeBon_mind} that the conscious personality of the individual in a crowd is submerged and that the collective crowd mind dominates; crowd behaviour is unanimous, emotional, and intellectually weak. 
In the second half of the twentieth century, research was focused on social behaviour in crowded situations, by studying e.g. emergency evacuations and the relation to the corresponding domain layout. 
In the 1970s, analytical formulas for crowd phenomena were derived from empirical data. 
The following decade, a split in the research activities occurred: experimental work was joined by studies aided by technology (e.g., computer vision) and computational simulations for graphic applications. 
In this era, simulations evolved from providing basic numerical data outputs to complex three-dimensional virtual environments.

In recent years, technological and scientific advancements have enabled real-life high-accuracy measurements of pedestrian trajectory data. 
The usage of overhead depth-sensing cameras \cite{Xovis} allows for the anonymous, large-scale acquisition of pedestrian trajectories without compromising quality or privacy. 
Before the arrival of such data, only qualitative models of pedestrian behaviour were available, but these developments have enabled research on quantitive models. 
Large-scale trajectory data opens up new possibilities for research on statistical descriptions of pedestrian ensembles, but many other applications have already been published \cite{CorbTosc_1,CorbTosc_2,CorbTosc_3} as well.

In the current age, scientific works range from understanding single pedestrian behaviour to dynamic crowd interactions. 
The COVID-19 pandemic has proven that human movements are of extreme relevance for modern society as well \cite{physical_distance}. 
Data collection methods mainly include real-life field observations, controlled experiments, survey-based methods and pedestrian simulation approaches. 
Recent work shows enormous potential for data collection methods, but restrictions are present as well. 
For example, field observations are limited by privacy-related issues and controlled experiments often fail to realistically represent real-life scenarios.
Research on pedestrian dynamics is characterized by a very large heterogeneity in published works. 
This is caused by a large range of science branches that research pedestrian dynamics, including computer science, engineering, mathematics, physics, psychology, and social science. 
Additionally, different works aim to study different phenomena, e.g., the emergence of crowd self-organization, vibrations in bridges caused by walking crowds or emergency evacuations.

To streamline further discussions on pedestrian dynamics, it is beneficial to introduce some definitions on topics presented in this thesis. 
Currently, there is some disagreement on literature definitions due to great heterogeneity in published works. In 2019, the Consortium for the Physics and Psychology of Human Crowd Dynamics, constituted a glossary of terms related to crowd research \cite{glossary_for_pedestrian}. Their work is not presented as absolute truth on formal definitions but reflects current views and used interpretations of crowd-related terminology. These definitions will be used as a guideline in this thesis as well. In the glossary, a pedestrian is defined as a person moving on foot in a publicly accessible area. Further refinement of different pedestrian types is possible by including their motivation, such as pedestrian-commuter, pedestrian-shopper, or pedestrian-traveler. There is no clear agreement whether motionless persons should be still considered pedestrians, but nonmoving persons are still considered pedestrians in this thesis.



\end{document}
