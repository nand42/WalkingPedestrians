%%
%% Author: Dario Chinelli
%% begin 2019-12-04
%% last mod 2022-02-02
%%

% Preamble
\documentclass[class=article, crop=false]{standalone}

% Packages
\usepackage[subpreambles=true]{standalone}
\usepackage{import}
\usepackage{graphicx}
\usepackage{amsmath}
\usepackage{subfig}

%	References:
%	> Markov chains  from theory to implementation and experimentation by Gagniuc, Paul A

% Document
\begin{document}
This section's scope is to introduce some nomenclature and theoretic information for the following chapters.

\subsection{Markov Chain} % \label{chap:MarkovChain}
	\import{sections/}{markov_chain}
	

% ---
\FloatBarrier
\subsection{The model's name and scheme}
The aim of this section is to give the notation used about the models and the mathematical tools.

\paragraph{Model's name} (FAI DISEGNI)
The D2Q9 is the abbreviation for the "two dimensional space" (D2) and considering the "nine next near cells" (Q9).
The name used here of the first considered model is "\emph{D2Q9-model}" and it's also the simplest.
The second use the same components of the previous but including also the "nine previous near cells" so that a second (Q9) is added to the end; its name is "\emph{D2Q9Q9-model}".
The third model is again similar to the very first one and it also considers "time" (T); its reference name is than "\emph{TD2Q9-model}".
The fourth model, and last in this work, is named as "\emph{TD2Q9Q9-model}", so that this model is developed in a "two dimensional space" considering the "nine next near cells" and the "nine previous near cells".


\paragraph{The developed mathematical framework}
The mathematical framework MF utilized in this work is primarily composed by a tensor of probabilities; that immediately follows from the MC theory.
The reference name to this tensor in the following pages is $A$, where are added indexes, as subscripts.

For the simplest model used here, the \emph{D2Q9-model}, it's a three dimensional tensor named $A_{x y k}$.
For the second, the \emph{D2Q9Q9-model}, it's a four dimensional tensor named $A_{x y k h}$.
For the third, the \emph{TD2Q9-model}, it's a four dimensional tensor named $A_{t x y k}$.
For the fourth, the \emph{TD2Q9Q9-model}, it's a five dimensional tensor named $A_{t x y k h}$.

% ---
% \FloatBarrier
% \subsection{Cellular-Automata}
% 	\import{sections/}{cellular_automata}

% ---
% \subsection{Lattice-Boltzman}
%	\import{sections/}{lattice_boltzman}


\end{document}
