%%
%% Author: Dario Chinelli
%% begin 2019-12-04
%% last mod 2022-02-02
%%

% Preamble
\documentclass[class=article, crop=false]{standalone}

% Packages
\usepackage[subpreambles=true]{standalone}
\usepackage{import}
\usepackage{graphicx}
\usepackage{amsmath}



% Document
\begin{document}
The aim of this work is to clarify the possibility to analyze real life datas and to generate simulated dynamics of the pedestrian crowd.
To do so it's necessary to be able to collect datas from a real world situation with enough precision and high acquisition ratio.
Once datas are collected properly is then possible to \emph{learn} from it.
The dynamics of a pedestrian is a complex motion in which multiple conditions and forces play a role.
The motion of a single pedestrian in a crowd is a similarly complex problem.
Despite everywhere in the world it is possible to find and watch walking pedestrian, it is however not as simple to acquire data about their motion.
So the first problematic issue is the data acquisition.
A possible solution to this is the video recording at a given spot.
However this choice lead to others problems such us privacy violation and the object tracking from the video.
The Xovis sensor was used to collect the data showed in this work, this type of sensor is capable to solve both the problems above.
During the lasts decades the development of machine learning and imaging recognition has provided more tools to analyze this type of data.
This technological advancement have enabled real-life high-accuracy measurements of pedestrian trajectory directly in loco.
The data are acquired through the usage of overhead depth-sensing cameras.
This approach allows a large scale anonymous acquisition of pedestrian trajectories without compromising quality or privacy.
In this research a statistical approach is used to assimilate the average paths of pedestrians trajectories.
Based on this, four models are being studied to evaluate which one is better predicting the most probable path.
Due to the statistical approach this potential is also a probabilistic model, that make, or not, possible a good prediction based on probabilities.
The \emph{probability} derivates from the real-data observation.


\subsection{Cellular Automata Model}
Cellular automata (CA) belong to the family of discretised modelling approaches. The model consists of a discrete spacetime lattice, along with computational capabilities that govern the evolution of the model through space and time. CA approaches often feature a finite amount of physical states per lattice site, but this is not a requirement. CA models are characterised by two main features: locality, ensuring that interactions can only take place between a given set of neighbouring cells, and modularity, which requires every lattice cell to be an independent process. The latter renders CA approaches very suitable for parallelised computing.

In the context of pedestrian dynamics, CA models discretise the pedestrian domain into a grid of cells, where every cell hols information and the presence and walking direction of pedestrians. Cells can also be flagged to be not accessible, to model boundary conditions in the form of objects and obstacles. The model should also have a set of transition rules, governing pedestrian movements between different cells. Such rules are often defined by probabilities and stochastic choice models, hence the close connection with the stochastic modelling category.

Cellular automata were first applied successfully in the context of pedestrian dynamics by Blue and Adler in 1998 \cite{microsimulation_1}, simulating one-dimensional pedestrian traffic, which was later extended to two- dimensional traffic flows \cite{microsimulation_2}. CA models have also been applied successfully in the context of evacuation problems \cite{evacuation_1,evacuation_2} and in junction with with other modelling categories.

Much of the criticism towards CA-based approaches follows from the method’s discrete nature. Since the space-time lattices are often very symmetric, the lattices are considered to be too symmetric for realistic movements. Moreover, the finite number of states and rules per lattice cell cause non- natural homogeneous behaviour, as demonstrated by Bierlaire et al. \cite{criticism_1}. Approaches to overcome these limitations have been proposed by Lubas et al. \cite{criticism_2}, in which the authors created a non-homogeneous and asynchronous CA model with cell-dependent transition rules. Still, the CA model remains a popular platform for studying pedestrian dynamics following its computational simplicity.


\subsection{Data-Driven Model}
The data-driven category distinguishes itself by a strong dependence on real-life measured pedestrian behaviour. In the literature review, two different approaches are concerned, namely data-in-the-loop approaches and data-in-the-model approaches.
In the data-in-the-loop models, real pedestrian data(consisting of group behaviour or individual trajectories) are assembled into a collection, which is then used to perform simulations. In Lerner et al. (2007), pedestrian trajectories are captured from video recordings, which are used to generate natural pedestrian behaviour in a virtual environment \cite{crowds}. In the work by Porzycki (2014), a pedestrian simulation is coupled with a measurement setup, asdetected pedestrians are initialized as embodied agents in the simulation \cite{data_driven}. In 2010, Ju et al. introduced a crowd generation approach, in which crowd formations and individual trajectories were taken from video recordings \cite{mor_crowds}. These measurements were then used to create virtual interpolated crowds of different densities. All data-in-the-loop approaches suffer from interpolation artefacts causing non-realistic behaviour, especially in the limit of high densities.

Data-in-the-model are similar to the aforementioned methods, but have one key difference in their workings: the parameters of an existing simulation model are adjusted based on real pedestrian measurements. This category has much more works reported in the literature review, for all modelling categories considered, such as mechanical, cellular automata and stochastic models \cite{crowd_simulation,interactive_crowd_simulation,data_driven_simulation}. The most travails are encountered in the area of data extraction: it is time-consuming work to capture high-quality pedestrian measurements, moreover for large crowds.



\section{Challenges}

\paragraph{Starting from real data how can we define a good model to simulate a pedestrian in the crowd flow?}
In this type of system there is a multitude of \emph{forces} that determinate the path of a single pedestrian.
So let's take into account a single pedestrian $P$ that walks in a certain space.
The first type of interaction is the structure where $P$ can or cannot walk thought, that is defined as the whole domain $\Omega$.
The second interaction is between $P$ and the other pedestrians.
Every pedestrian needs a personal space all around, that is variable due the circumstance and it is not easy to be analytically determinate.
A third type of interaction are random events along the $P$'s path, real world events.
...

\paragraph{How to visualise the pedestrian's path using a multi-dimensional histogram?}
It is possible to plot every single trajectory, but this lead to a chaotic data representation and not so functional nor readable.
It is also easily possible to plot the \emph{heatmap} of a dataset to analyse the most "walked" areas.
Even if this second plot choice can takes into account more trajectories than the first and still be readable, it has a problem.
This second lead to a representation where the time dependency is completely lost.
...





\section{Relevance}


















\end{document}
