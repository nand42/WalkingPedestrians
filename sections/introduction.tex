%%
%% Author: Dario Chinelli
%% begin 2019-12-04
%% last mod 2022-02-02
%%

% Preamble
\documentclass[class=article, crop=false]{standalone}

% Packages
\usepackage[subpreambles=true]{standalone}
\usepackage{import}
\usepackage{graphicx}
\usepackage{amsmath}



% Document
\begin{document}
The aim of this work is to clarify the possibility to analyze real life datasets and simulate the pedestrian crowd starting from the Lattice model.

\section{Assimilating pedestrian dynamics}
The dynamics of pedestrians is essentially a chaotic motion in witch multiple conditions and forces are applied to the system.
The motion of a single pedestrian in a crowd is a similarly complex problem.
Since everywhere in the world is possible to find and watch walking pedestrian, is not as simple to acquire data about their motion.
So the first problematic issue is the data acquisition.
A possible, and first, solution to this is the video recording of a spot.
However this choice lead to others problem such us privacy leak and the object tracking from the video.
Still the first issue doesn't have a real solution except if faces are blurred.
During the lasts decades the development of machine learning and imaging recognition has provided more tools to analyze this type of data.
So that is now possible to elaborate the image of a video surveillance system and obtain analytic datas.
Other recent technological advancements have also enabled real-life high-accuracy measurements of pedestrian trajectory.
The data are acquired through the usage of overhead depth-sensing cameras.
This second approach allows a large scale anonymous acquisition of pedestrian trajectories without compromising quality or privacy.
In this research a statistical approach is used to assimilate the average paths of pedestrians trajectories.
Based on this, four models are being studied to evaluate which is better in prediction.
The assumption is that a crowd produces an \emph{effective potential}. 
Due to the statistical approach this potential is also a probabilistic model, that make, or not, possible a good prediction based on probabilities.
The \emph{probability} is inducted by the real-data observation.
The path of synthetic pedestrian is given by a Monte Carlo simulation that defines the probability to move in every direction given a position.
To make this study possible the space and time discretization is essential.



\section{Challenges}

\paragraph{Starting from real data how can we define a good model to simulate a pedestrian in the crowd flow?}
In this type of system there is a multitude of \emph{forces} that determinate the path of a single pedestrian.
So let's take into account a single pedestrian $P$ that walks in a certain space.
The first type of interaction is the structure where $P$ can or cannot walk thought, that is defined as the whole domain $\Omega$.
The second interaction is between $P$ and the other pedestrians.
Every pedestrian needs a personal space all around, that is variable due the circumstance and it is not easy to be analytically determinate.
A third type of interaction are random events along the $P$'s path, real world events.
...

\paragraph{How to visualise the pedestrian's path using a multi-dimensional histogram?}
It is possible to plot every single trajectory, but this lead to a chaotic data representation and not so functional nor readable.
It is also easily possible to plot the \emph{heatmap} of a dataset to analyse the most "walked" areas.
Even if this second plot choice can takes into account more trajectories than the first and still be readable, it has a problem.
This second lead to a representation where the time dependency is completely lost.
...





\section{Relevance}


















\end{document}
