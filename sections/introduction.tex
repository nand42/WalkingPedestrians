%%
%% Author: Dario Chinelli
%% begin 2019-12-04
%% last mod 2022-02-02
%%

% Preamble
\documentclass[class=article, crop=false]{standalone}

% Packages
\usepackage[subpreambles=true]{standalone}
\usepackage{import}
\usepackage{graphicx}
\usepackage{amsmath}



% Document
\begin{document}
The aim of this work is to clarify the possibility to analyze real life datasets and simulate the pedestrian crowd starting from the Lattice model.

\section{Assimilating pedestrian dynamics}




\section{Challenges}

\paragraph{Starting from real data how can we define a good model to simulate a pedestrian in the crowd flow?}
In this type of system there is a multitude of \emph{forces} that determinate the path of a single pedestrian.
So let's take into account a single pedestrian $P$ that walks in a certain space.
The first type of interaction is the structure where $P$ can or cannot walk thought, that is defined as the whole domain $\Omega$.
The second interaction is between $P$ and the other pedestrians.
Every pedestrian needs a personal space all around, that is variable due the circumstance and it is not easy to be analytically determinate.
A third type of interaction are random events along the $P$'s path, real world events.
...

\paragraph{How to visualise the pedestrian's path using a multi-dimensional histogram?}
It is possible to plot every single trajectory, but this lead to a chaotic data representation and not so functional nor readable.
It is also easily possible to plot the \emph{heatmap} of a dataset to analyse the most "walked" areas.
Even if this second plot choice can takes into account more trajectories than the first and still be readable, it has a problem.
This second lead to a representation where the time dependency is completely lost.
...





\section{Relevance}


















\end{document}
