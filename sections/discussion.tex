%%
%% Author: Dario Chinelli
%% begin 2019-12-04
%% last mod 2022-02-02
%%

% Preamble
\documentclass[class=article, crop=false]{standalone}

% Packages
\usepackage[subpreambles=true]{standalone}
\usepackage{import}
\usepackage{graphicx}
\usepackage{amsmath}

% Document
\begin{document}
%Discussion/Conclusion documentation here

This work is leaded by a statistical approach, which is possible thanks to the access at a large quantity of data.
Its focus was to generate a model, a mathematical framework, that learns the probability distribution based on real world situation.
A total of four different models have been studied, during the period of study in Eindhoven, and are presented in Chapter 2.
The results in Chapter 3 were given generating the dynamic simulation for each model, finding similar paths to the real original data.
The approach in this thesis is applied to the pedestrian problem and, for a statistical and data-driven point of view, it is really capable in learning the behaviour of those paths.
The technological tools developed to make this work possible are themself a positive result for this thesis.

This work has proven that the concept of a statistical approach to this problem may lead to positives results and applications.
Firstly, the dynamic simulation generate pedestrians that may move only where real data have moved through.
So it is not necessary to implement boundary conditions, excepts for borders' map, because the framework does it \emph{by construction}.
This concept has a deep and strong impact on the study's results and it also simplify the model itself.
Secondly, it was shown that this tools may be applied to a great number of information and still running on an average portable computer.
So that it opens the possibilities to in-loco solutions for new experiments.
Of course, by increasing the number of data and the complexity of the model it would require a greater amount of calculation capacity.

In future works, the same approach may be applied to other systems, that require the same statistical and data-driven point of view: 
to study the satellites' orbit around Earth, the paths given by GPS datas may be easily analyzed, it's also possible the study the path given by the neurons activity.
In general, this solution is applicable where a complex system is nor deterministic nor ordinated and the statistical approach may lead to a solution or a part of it.



\end{document}
