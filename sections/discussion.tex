%%
%% Author: Dario Chinelli
%% begin 2019-12-04
%% last mod 2022-02-02
%%

% Preamble
\documentclass[class=article, crop=false]{standalone}

% Packages
\usepackage[subpreambles=true]{standalone}
\usepackage{import}
\usepackage{graphicx}
\usepackage{amsmath}

% Document
\begin{document}
%Discussion/Conclusion documentation here

This work is led by a statistical approach, which is possible thanks to access to a large quantity of data.
Its focus is to create a model, that learns the probability distribution based on a real-world situation.
The challenge was to analyze and solve the \emph{pedestrian problem}: determinate what is the pedestrian's behaviour when in a crowd of people and so which is its most probable path.
\\To do so, in Chapter 2, a total of four different models have been presented and studied in this work.
The result by making a statistical study on this matter is probabilistic and consequently, given a starting condition, it's not possible to determinate a single solution but a group of solutions.
Choosing the most probable path gives the solution to the challenge, which leads to the understanding of how a pedestrian interacts with a crowd of people.
\\The results in Chapter 3 were given generating the dynamic simulation for each model, finding similar paths to the real original data.
The dataset consists of a variety of trajectories and the data is collected over an entire daytime.
The focus of this thesis is the analysis of paths, without taking into account the density of the crowd, but the pedestrian's trajectories.

During my thesis work, made possible by the \emph{Crowdflow Research Group} at TU/e in Eindhoven, I participated during some part of the data collection, giving a concrete contribution to this work.
\\Most of my work was about defining the physical model and writing code \cite{repoPIA} to analyze the data and generate dynamic simulations.
The code is entirely written in Python with an object-oriented approach, built as a Python module;
it takes the CSV files as input, analyzes the data and learns the transition matrices, generates the simulations and plots different types of views.

The approach in this thesis is applied to the pedestrian problem and, from a statistical and data-driven point of view, it is capable of learning the behaviour of those paths.
The technological tools developed to make this work possible are themself a positive result for this thesis.
\\This work has proven that the concept of a statistical approach to this problem may lead to positives results and applications.
Firstly, the dynamic simulation generates pedestrians that may move only where real data have moved through.
So it is not necessary to implement boundary conditions, except for the borders' map, because the framework does it \emph{by construction}.
This concept has a deep and strong impact on the study's results and it also simplifies the model itself.
Secondly, it was shown that these tools may be applied to a great number of information and still run on an average portable computer.
So that it opens the possibilities to in-loco solutions for new experiments.
Of course, increasing the number of data and the complexity of the model would require a greater amount of calculation capacity.

In future works, the same approach may be applied to other systems, that require the same statistical and data-driven point of view: 
to study the satellites' orbit around Earth, the paths given by GPS data may be easily analyzed, it's also possible to study the path given by the neurons' activity.
In general, this solution is applicable where a complex system is neither deterministic nor ordinated and the statistical approach may lead to a solution or a part of it.



\end{document}
