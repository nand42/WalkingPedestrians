%%
%% Author: Dario Chinelli
%% begin 2019-12-04
%% last mod 2022-02-02
%%

% Preamble
\documentclass[class=article, crop=false]{standalone}

% Packages
\usepackage[subpreambles=true]{standalone}
\usepackage{import}
\usepackage{graphicx}
\usepackage{amsmath}

% Document
\begin{document}
%Discussion/Conclusion documentation here

This work is leaded by a strong statistical approach, which is possible thanks to the large scale of data.
The focus was to be able to generate a model, a mathematical framework, that learns the probability distribution based on real world situation.
Four different models have been studied and presented in Chapter 2.
The results were given generating dynamic simulations similar to the real original data.
The approach in this thesis is applied to the pedestrian problem and, for a statistical and data-driven point of view, it is really capable in learning the behaviour of those paths.
The technological tools developed to make this work possible are themself a positive result for this thesis.
It was shown that this tool may be applied to a great number of information and still running on an average portable computer.
Of course, by increasing the number of data and the complexity of the model it would require a greater amount of calculation capacity.

In future works, the same approach may be applied to other systems, that require the same statistical and data-driven point of view: 
to study the satellites' orbit around Earth, the paths given by GPS datas may easily analyzed, it's also possible the study the path given by the neurons activity, 
or in general where a complex system is nor deterministic nor ordinated the statistical approach may lead to a solution or a part of it.



\end{document}
