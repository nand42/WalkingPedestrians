%%
%% Author: Dario Chinelli
%% begin 2019-12-04
%% last mod 2022-02-21
%%

% Preamble
\documentclass[class=article, crop=false]{standalone}

% Packages
\usepackage[subpreambles=true]{standalone}
\usepackage{import}
\usepackage{graphicx}
\usepackage{amsmath}

% Document
\begin{document}
In this study there is a total of four models: \emph{D2Q9}, \emph{D2Q9Q9}, \emph{TD2Q9}, \emph{TD2Q9Q9}.
The firsts two are only dependent by the position in space, also called \emph{time-independents}.
The others two are dependent by the position and time, also called \emph{time-dependents}.
Whereas there is also a distinction between the D2Q9s and the D2Q9Q9s.
For the D2Q9s what it's doing is considering the velocity from a cell to another, so just the change in position.
For the D2Q9Q9s it's also considering the acceleration, so the change in velocity.
The starting point of each one is the dataset, collected form a real life situation.
Since each of them are entirely based on real world pedestrian's path in a crowd, those models simulates an \emph{effective potential} (EP).
This potential considers the imposed limit due to the presence of others pedestrians, such as pedestrians tend to not collide each others.
It also considers the boundary condition given by the structural environment.
The strong point of this EP is that is generated by the real world observation and not built by hand.
With the aim of reproducing realistic pedestrians movements, synthetic paths are created from the models.
Every model generate one trajectory that simulate just one pedestrian in a statistical crowd.
When simulating more paths it consider pedestrian that walks alone in the crowd.
This model doesn't consider the interaction made by the others simulated pedestrians.

\paragraph{Notation} Lets assume $\gamma=\gamma( \vec x_c)$ a pedestrian's path, where $\vec x_c = (x_c, y_c)$ has a bi-dimensional spacial dependancy.
Given a field $\Omega_c$, the continuous space where pedestrians are tracked, the path $\gamma$ in that space has a start position $A$ and an end position $B$.
The field $\Omega$ is than divided into \emph{rectangular} cells, dividing the real space along $x$, whit maximum extension indicated as $L_x$, in a certain number of cells $D_x$; 
as well for the $y$-direction, with obvious notation: $L_y$ and $D_y$.
After this discretization is obtained a \emph{grid space} $\Omega_g$.
Where every path $\gamma$ is converted from continuous $\gamma=\gamma(x_c, y_c)$ to discrete coordinate $\gamma=\gamma(x_g, y_g)$, referred to the \emph{grid}.
To lighten up the notation when speaking of \emph{grid space} it is simply used $(x, y)$ in reference to the discrete grid position.

\paragraph{The standard D2Q9 configuration} 
In reference to the (Figure \ref{fig:D2Q9_k}).
This \emph{map} is set for each position $(x_0, y_0)$ in the grid space and it represents the eight neighbors and the central position where a pedestrian could go.
Each direction will be associated to a certain transition probability.
\import{draw/}{D2Q9_directions}
When a trajectory change position, in the grid space (Figure \ref{fig:D2Q9_c}), from $P_0=(x_0, y_0)$ to $P_1=(x_1, y_1)$ is associated a transition.
The transition is identified by a number $k = 0,1,...,8$ such that is unique.
It is derived from the series of coordinates for each trajectory and each step in time.
When the calculation is made for each step, for every position in time is also associated a transition number, that represents where is going to go in the next step.
\import{draw/}{D2Q9_directions_coordinates}
If this transition is associated to the change in position it identify a certain velocity, as vector, with a certain direction.
Iterating this procedure to the entire pedestrian's trajectory it is possible to get something like what's illustrated in the (Figure \ref{illustrate_eg_path}).
In that figure it is possible to distinguish the path in the continuous space and the discrete path in the grid space.
It also shows the direction of the next movement for each position with arrows that are consistent with the velocity arrows in each position.
The numbers are the value of the $k$-index in each position, it is solid with the maps above.
This lead the discussion directly to the first model $D2Q9$ in the next paragraph.
\import{draw/}{illustrate_eg_path}



%% --- SECTION SEPARATOR---

\subsection{Model D2Q9} \label{chap:Model_D2Q9}
The simplest model considered here is called $D2Q9-model$.
This model is a time-independent and it consider the velocity of the pedestrian.
Given a starting position $(x_0, y_0)$ in the field $\Omega$.
It uses the nine closest possible positions where a pedestrian could go from that point.
With the $D2Q9-model$ is than possible to know, for each position $(x_0, y_0)$, the probability to go up, down, left, right or a combination of those movements.
\paragraph{Transitions} from the initial position $P_0=(x_0, y_0)$ to the next closest cell in the grid $P_k$ are defined by the index $k$.
So that the index $k$ gives the direction of the transition.
For instance a transition from $P_0$ to $P_1$ and a transition from $P_0$ to $P_2$ are defined by:
\begin{equation*}
\begin{split}
P_0 &\to P_1 \\
(x_0, y_0) &\to (x_0+1, y_0)
\end{split}\quad\quad\quad
\begin{split}
P_0 &\to P_2 \\
(x_0, y_0) &\to (x_0, y_0+1)
\end{split}
\end{equation*}
% D2Q9 diagram

\import{draw/}{D2Q9_directions_arrows}
Considering the (Figure \ref{fig:D2Q9}) all the transitions are associated to a specific $k$.
This is a particular Markov Chain (see \ref{chap:MarkovChain} ) where there are a total of \emph{nine} states.
Between these states the transitions always and only start from the $P_0$ state to go to the others $P_k$ states or itself.
The same concept is graphically represented with the diagram in (Figure \ref{fig:diagram_MC_D2Q9}).
To every transition is associated a certain probability to happen.
Formally this probability is given by the initial and the final states: $p_{if}$.
Since there always is the same starting state, it is possible to omit it.
So that the probability of the transition from $P_0$ to $P_k$ is expressed by $p_k$, where the index $k$ points to the ending state.
It means that for each position in $\Omega$ it's possible to say how likely is to "step forward" or "turn right" and so on.
Then, once in the new position, it's again possible to say the most probable direction that the pedestrian will choose.
The same prediction is applicable to the whole space, mapped by the real datas.
\import{draw/}{D2Q9_MarkovDiagram}
With this structure it is then possible to create a tensor $A$ with three indices, or more for the other models.
Taking into account the simplest model, as above, the tensor is $A_{x y k}$.
Where every entries is the probability $p$ to move along the $k$ direction from the location $(x, y)$.
Since the aim of every models is to simulate a pedestrian in the crowd, this tensor is the key to get to the result.


\subsection{Model D2Q9Q9}
The conceptual step forward of the study is to not only consider the next position, but also the previous.
Hence given a trajectory $\gamma$ in the grid space of a pedestrian that make a transition for each time step.
For each point $P_0$ of $\gamma$ it is possible to determinate where it was before at $P_{-1}$ and where is going to be after at $P_{+1}$.
The index that represents the \emph{next} position is $k$, meanwhile the index that represents the \emph{previous} position is $h$.
For instance it is given the table of the coordinates and the two indexes related to the (Figure \ref{fig:diagram_MC_D2Q9}) in the  (Table \ref{tab:diagram_MC_D2Q9})


\begin{table}[h!]
\centering
\label{tab:diagram_MC_D2Q9}
\begin{tabular}{|c|c|c|c|c|}
\hline
Time step & $x_g$ & $y_g$ & $k$-index & $h$-index  \\ \hline
1         & 5 & 1 & 3 & 0 \\ \hline
2         & 4 & 1 & 2 & 1 \\ \hline
3         & 4 & 2 & 3 & 4 \\ \hline
4         & 3 & 2 & 2 & 1 \\ \hline
5         & 3 & 3 & 5 & 4 \\ \hline
6         & 4 & 4 & 2 & 7 \\ \hline
7         & 4 & 5 & 2 & 4 \\ \hline
8         & 4 & 6 & 2 & 4 \\ \hline
9         & 4 & 7 & 2 & 4 \\ \hline
10        & 4 & 8 & 6 & 4 \\ \hline
11        & 3 & 9 & 0 & 8 \\ \hline
\end{tabular}
\captionsetup{width=.6\linewidth}
\caption{This is tabulated the trajectory of the same illustrative pedestrian as above in (Figure \ref{fig:diagram_MC_D2Q9}).
Here is expressed the position time to time, the index of the following move and the index of the previous move.}
\end{table}







\subsection{Model TD2Q9}








\subsection{Model TD2Q9Q9}












\end{document}
