%%
%% Author: Dario Chinelli
%% begin 2019-12-04
%% last mod 2022-02-02
%%

% Preamble
\documentclass[class=article, crop=false]{standalone}

% Packages
\usepackage{tikz}
\usetikzlibrary{positioning}

% Document
\begin{document}



\begin{figure}[h!]
\centering
\begin{tikzpicture}[->, >=stealth', auto, semithick, node distance=1.6cm]
\tikzstyle{every state}=[fill=white, draw=black, thick, text=black, scale=1]
\node[state]    (A)		   	{$P_0$};
\node[state]    (E)[above of=A]   {$P_4$};
\node[state]    (D)[left of=E]   	{$P_3$};
\node[state]    (C)[left of=D]   	{$P_2$};
\node[state]    (B)[left of=C]   	{$P_1$};
\node[state]    (F)[right of=E]   	{$P_5$};
\node[state]    (G)[right of=F]   	{$P_6$};
\node[state]    (H)[right of=G]   	{$P_7$};
\node[state]    (I)[right of=H]   	{$P_8$};
\path
(A) edge[loop below]  		node{$p_0$}  	(A)
(A) edge[bend left, below] 		node{$p_1$}     (B)
(A) edge[bend left, below] 		node{$p_2$}     (C)
(A) edge[bend left, below]      	node{$p_3$}     (D)
(A) edge[bend left, below]      	node{$p_4$}     (E)
(A) edge[bend right, below]      	node{$p_5$}     (F)
(A) edge[bend right, below]      	node{$p_6$}     (G)
(A) edge[bend right, below]      	node{$p_7$}     (H)
(A) edge[bend right, below]      	node{$p_8$}     (I);
\end{tikzpicture}
\captionsetup{width=.8\linewidth}
\caption{The Markov Chain diagram of the system. 
The states are indicated with circles and labeled with $P_k$. 
The transitions are indicated with arrows and labeled with theirs probability $p_k$}
\label{fig:diagram_MC_D2Q9}
\end{figure}




\end{document}
